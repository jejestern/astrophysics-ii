% Packages
\RequirePackage{tensor}
\RequirePackage{physics}
\RequirePackage[
      per-mode = reciprocal,
      exponent-product = \cdot,  
]{siunitx}

% trigonometry
\DeclareMathOperator{\arcosh}{arcosh}
\DeclareMathOperator{\arsinh}{arsinh}
\DeclareMathOperator{\artanh}{artanh}
\DeclareMathOperator{\arsech}{arsech}
\DeclareMathOperator{\arcsch}{arcsch}
\DeclareMathOperator{\arcoth}{arcoth}

% topology
\providecommand{\closure}[1]{ \overline{#1} }
\providecommand{\boundary}[1]{ \partial #1 }
\providecommand{\interior}[1]{ {#1}^\circ }

% vector
\renewcommand{\vec}[1]{\symbf{#1}}

% vector calculus
\newcommand{\Lap}{ \vec{\Delta} }
\newcommand{\Rot}{ \vec{\nabla} \times }
\newcommand{\Div}{ \vec{\nabla} \cdot }
\newcommand{\Grad}{ \vec{\nabla} }
\newcommand{\Dal}{ \square }

% number sets
\providecommand{\R}{\mathbb{R}}
\providecommand{\N}{\mathbb{N}}
\providecommand{\Z}{\mathbb{Z}}
\providecommand{\Nz}{\mathbb{N}_0}
\providecommand{\Cz}{ \mathbb{C}^* }
% \renewcommand{\C}{\mathbb{C}}
\providecommand{\CC}{\mathbb{C}}

% automatic paired delimiters from mathtools
\let\norm\relax %relax norm from the physics package
\DeclarePairedDelimiter{\norm}{\lVert}{\rVert}

\let\abs\relax %relax abs from the physics package
\DeclarePairedDelimiter{\abs}{\vert}{\vert}

\DeclarePairedDelimiter{\floor}{\lfloor}{\rfloor}
\DeclarePairedDelimiter{\ceil}{\lceil}{\rceil}


% lie groups and algebras
\providecommand{\lie}[1]{\mathfrak{#1}}
\DeclareMathOperator{\Lie}{Lie}
\newcommand{\com}[2]{\left[ #1, #2 \right]}

% lie groups
\newcommand{\gl}{\lie{gl}}
\renewcommand{\sl}{\lie{sl}}
\renewcommand{\o}{\lie{o}}
\newcommand{\so}{\lie{so}}
\renewcommand{\u}{\lie{u}}
\newcommand{\su}{\lie{su}}

% inner product
\providecommand{\iprod}[2]{\left\langle #1 , #2 \right\rangle}


% integrals
\providecommand{\dint}[4]{
	\int_{#1}^{#2} #3 \, \dd{#4}
}

\providecommand{\Int}[3]{
	\int_{#1}^{#2} #3
}


% adjoint
\DeclareMathOperator{\adj}{adj}

% function restriction
\providecommand{\restrict}[2]{
	\left.\kern-\nulldelimiterspace % automatically resize the bar with \right
	#1 % the function
	\vphantom{|} % pretend it's a little taller at normal size
	\right|_{#2} % this is the delimiter
}

% limit
\providecommand{\limit}[2]{
	\lim_{#1 \to #2}
}


% case for cases envrionment
\providecommand{\caseif}[2]{ {#1} &\text{if } {#2} }
\providecommand{\caseelse}[1]{ {#1} &\text{else}}



% complex conjugate
\newcommand{\cc}[1]{\widebar{#1}}
\newcommand{\ccs}[1]{\bar{#1}}  % nice for single letter
\newcommand{\ccm}[1]{\widebar{#1}}  % nice for multiple letters


% maps
\DeclareMathOperator{\id}{\mathbbm{1}} % identity
\DeclareMathOperator{\im}{im} % image
\DeclareMathOperator{\nullity}{null} 
\providecommand{\class}[1]{\left[ {#1} \right]}	 % equivalence class
\DeclareMathOperator{\supp}{supp} % support
\DeclareMathOperator{\diag}{diag}

%quotient space
\providecommand{\qs}[2]{
   	\mathchoice
   	{% \displaystyle
   		\text{\raise1ex\hbox{$#1$}\Big/\lower1ex\hbox{$#2$}}%
   	}
   	{% \textstyle
   		#1\,/\,#2
   	}
   	{% \scriptstyle
   		#1\,/\,#2
   	}
   	{% \scriptscriptstyle  
   		#1\,/\,#2
   	}
}


% special sets
\DeclareMathOperator{\Gl}{Gl}
\DeclareMathOperator{\End}{End}
\DeclareMathOperator{\Hom}{Hom}
\DeclareMathOperator{\Aut}{Aut}
\DeclareMathOperator{\Iso}{Iso}
\let\O\relax
\let\U\relax
\DeclareMathOperator{\Sp}{Sp}
\DeclareMathOperator{\O}{O}
\DeclareMathOperator{\SO}{SO}
\DeclareMathOperator{\U}{U}
\DeclareMathOperator{\SU}{SU}
\DeclareMathOperator{\SL}{SL}


% set algebra
\providecommand{\set}[1]{ \left\{ {#1} \right\} }
\providecommand{\Set}[2]{
	\left\{
	\, {#1} \, \middle| \, {#2} \,
	\right\}
}
\providecommand{\iso}{\cong}
\providecommand{\union}{\cup}
\providecommand{\inters}{\cap}
% disjoint
\providecommand{\dunion}{\sqcup}
\providecommand{\dinters}{\sqcap}
% big
\providecommand{\biginters}{\bigcap}
\providecommand{\bigunion}{\bigcup}
% big & disjoint
\providecommand{\bigdinters}{\bigsqcap}
\providecommand{\bigdunion}{\bigsqcup}



% tensor product
\newcommand{\tprod}{\otimes} % tensor product
\newcommand{\dsum}{\oplus} % direct sum


%Definition
\providecommand*{\defeq}{\mathrel{\vcenter{\baselineskip0.5ex \lineskiplimit0pt
			\hbox{\scriptsize.}\hbox{\scriptsize.}}}%
	=}
\providecommand*{\eqdef}{= \mathrel{\vcenter{\baselineskip0.5ex \lineskiplimit0pt
			\hbox{\scriptsize.}\hbox{\scriptsize.}}}%
	}


%
%
% define widebar, which might be better than overline
%
%

\makeatletter
\let\save@mathaccent\mathaccent
\newcommand*\if@single[3]{%
  \setbox0\hbox{${\mathaccent"0362{#1}}^H$}%
  \setbox2\hbox{${\mathaccent"0362{\kern0pt#1}}^H$}%
  \ifdim\ht0=\ht2 #3\else #2\fi
  }
%The bar will be moved to the right by a half of \macc@kerna, which is computed by amsmath:
\newcommand*\rel@kern[1]{\kern#1\dimexpr\macc@kerna}
%If there's a superscript following the bar, then no negative kern may follow the bar;
%an additional {} makes sure that the superscript is high enough in this case:
\newcommand*\widebar[1]{\@ifnextchar^{{\wide@bar{#1}{0}}}{\wide@bar{#1}{1}}}
%Use a separate algorithm for single symbols:
\newcommand*\wide@bar[2]{\if@single{#1}{\wide@bar@{#1}{#2}{1}}{\wide@bar@{#1}{#2}{2}}}
\newcommand*\wide@bar@[3]{%
  \begingroup
  \def\mathaccent##1##2{%
%Enable nesting of accents:
    \let\mathaccent\save@mathaccent
%If there's more than a single symbol, use the first character instead (see below):
    \if#32 \let\macc@nucleus\first@char \fi
%Determine the italic correction:
    \setbox\z@\hbox{$\macc@style{\macc@nucleus}_{}$}%
    \setbox\tw@\hbox{$\macc@style{\macc@nucleus}{}_{}$}%
    \dimen@\wd\tw@
    \advance\dimen@-\wd\z@
%Now \dimen@ is the italic correction of the symbol.
    \divide\dimen@ 3
    \@tempdima\wd\tw@
    \advance\@tempdima-\scriptspace
%Now \@tempdima is the width of the symbol.
    \divide\@tempdima 10
    \advance\dimen@-\@tempdima
%Now \dimen@ = (italic correction / 3) - (Breite / 10)
    \ifdim\dimen@>\z@ \dimen@0pt\fi
%The bar will be shortened in the case \dimen@<0 !
    \rel@kern{0.6}\kern-\dimen@
    \if#31
      \overline{\rel@kern{-0.6}\kern\dimen@\macc@nucleus\rel@kern{0.4}\kern\dimen@}%
      \advance\dimen@0.4\dimexpr\macc@kerna
%Place the combined final kern (-\dimen@) if it is >0 or if a superscript follows:
      \let\final@kern#2%
      \ifdim\dimen@<\z@ \let\final@kern1\fi
      \if\final@kern1 \kern-\dimen@\fi
    \else
      \overline{\rel@kern{-0.6}\kern\dimen@#1}%
    \fi
  }%
  \macc@depth\@ne
  \let\math@bgroup\@empty \let\math@egroup\macc@set@skewchar
  \mathsurround\z@ \frozen@everymath{\mathgroup\macc@group\relax}%
  \macc@set@skewchar\relax
  \let\mathaccentV\macc@nested@a
%The following initialises \macc@kerna and calls \mathaccent:
  \if#31
    \macc@nested@a\relax111{#1}%
  \else
%If the argument consists of more than one symbol, and if the first token is
%a letter, use that letter for the computations:
    \def\gobble@till@marker##1\endmarker{}%
    \futurelet\first@char\gobble@till@marker#1\endmarker
    \ifcat\noexpand\first@char A\else
      \def\first@char{}%
    \fi
    \macc@nested@a\relax111{\first@char}%
  \fi
  \endgroup
}
\makeatother











\endinput
